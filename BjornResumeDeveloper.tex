\documentclass[11pt,oneside]{article}
\usepackage{geometry}
\usepackage{ifthen}
%\usepackage[T1]{fontenc}

\pagestyle{empty}
\geometry{letterpaper,tmargin=0.75in,bmargin=0.75in,lmargin=0.75in,rmargin=0.75in,headheight=0in,headsep=0in,footskip=.3in}

\setlength{\parindent}{0in}
\setlength{\parskip}{0in}
\setlength{\itemsep}{0in}
\setlength{\topsep}{0in}
\setlength{\tabcolsep}{0in}

% Name and contact information
\newcommand{\name}{Bj\o rn Hansen}
\newcommand{\address}{16630 NE 92ND ST\\ Redmond, WA 98052}
\newcommand{\phone}{(206) 508-1235}
\newcommand{\email}{holomorph@users.sourceforge.net}

%%%%%%%%%%%%%%%%%%%%%%%%%%%%%%%%%%%%%%%%%%%%%%%%%%%%%%%%%
% New commands and environments

% This defines how the name looks
\newcommand{\bigname}[1]{
    \fontfamily{ppl}\selectfont\huge\scshape#1 \normalsize
}

% A ressection is a main section (<H1>Section</H1>)
\newenvironment{ressection}[1]{
	\vspace{4pt}
	{\fontfamily{phv}\selectfont\Large#1}
	\begin{itemize}
	\vspace{3pt}
}{
	\end{itemize}
}

% A resitem is a simple list element in a ressection (first level)
\newcommand{\resitem}[1]{
	\vspace{-4pt}
	\item \begin{flushleft} #1 \end{flushleft}
}

% A ressubitem is a simple list element in anything but a ressection (second level)
\newcommand{\ressubitem}[1]{
	\vspace{-1pt}
	\item \begin{flushleft} #1 \end{flushleft}
}

% A resbigitem is a complex list element for stuff like jobs and education:
%  Arg 1: Name of company or university
%  Arg 2: Location
%  Arg 3: Title and/or date range
\newcommand{\resbigitem}[3]{
	\vspace{-5pt}
	\item
	\ifthenelse{\equal{#3}{}}
	{\textbf{#1}---#2 } % third argument is empty
	{\textbf{#1}---#2 \\ \textit{#3} } 
}

% This is a list that comes with a resbigitem
\newenvironment{ressubsec}[3]{
	\resbigitem{#1}{#2}{#3}
	\vspace{-2pt}
	\begin{itemize}
}{
	\end{itemize}
}

% This is a simple sublist
\newenvironment{reslist}[1]{
	\resitem{\textbf{#1}}
	\vspace{-5pt}
	\begin{itemize}
}{
	\end{itemize}
}



%%%%%%%%%%%%%%%%%%%%%%%%%%%%%%%%%%%%%%%%%%%%%%%%%%%%%%%%%
% Now for the actual document:
\begin{document}
\begin{center}
\bigname{\name}
\rule{\textwidth}{1pt}
\email \\
\phone \\
\vspace{1pt}
\address
\end{center}

\begin{ressection}{EDUCATION}
    \begin{ressubsec}{University of Washington}{ Seattle, Washington, 2004 - 2007}
    {Master of Science in Aeronautics and Astronautics}
    \ressubitem{Depth Area of Study: Plasma Science}
	\end{ressubsec}
    \begin{ressubsec}{University of Victoria}{Victoria, British Columbia, 2000 - 2004}
    {Bachelor of Science}
    \resitem{Graduated with Distinction in June 2004 with a Major in Physics, and a Minor in Mathematics}
	\end{ressubsec}
\end{ressection}
\begin{ressection}{EXPERIENCE}
    \begin{ressubsec}{Volt at Microsoft, XBOX}{Redmond, Washington, May 2008 - Present}
    {Software Development in Test (SDET II)} 
    \ressubitem{Worked as part of the Xbox Foundation Team on development of the 'Test Case Scheduler' test harness, which runs automated test cases from Product Studio and auto-files bugs.  Also worked with team members on the UI automation technology.  Additionally responsible for keeping Test Case Scheduler running on the test benches in the lab and following up on issues found during test runs.}
    \ressubitem{Supervisor: Russ Christensen}
    \end{ressubsec}
    \begin{ressubsec}{Volt at Microsoft, Natural Language Group}{Redmond, Washington, August 2007 - May 2008}
    {Software Development in Test (SDET II)} 
    \ressubitem{Wrote code which tests and verifies functionality of proofing tools, primarily the speller engine.}
    \ressubitem{Supervisor: Xiaolan Xing}
    \end{ressubsec}
    \begin{ressubsec}{University of Washington, RPPL}{ Seattle, Washington, September 2004 - August, 2007}
    {Research assistant}
    \ressubitem{Worked on an Innovative Confinement Concept device for magnetically confined fusion plasmas at the Redmond
    Plasma Physics Laboratory. Primary projects included design, implementation, and documentation of the glow
    discharge system, heater control system, and the vacuum control system. The vacuum control system consists of
    a network of DeviceNet hardware devices, controlled by software written using LabView. Additional duties
    included: conducting tests, designing and selecting parts, machining components and coordinating with other
    scientists and engineers to develop designs and specifications.}
    \ressubitem{Supervisor: Dr. Alan Hoffman}
	\end{ressubsec}
\end{ressection}
\begin{ressection}{SKILLS}
    \begin{reslist}{ Languages}
\ressubitem{C and C++, Python, Bash, C\#, Java, Javascript, Perl, SQL}
    \end{reslist}
    \begin{reslist}{ Software}
\ressubitem{Development Environments: Visual Studio, Eclipse, Dev C++, Kdevelop, Code::Bocks, Emacs, SPE}
\ressubitem{Debugging: gdb, pdb, Visual Studio Debugger, kd}
\ressubitem{Testing Frameworks: Junit, pyunit, Motif, Nunit}
\ressubitem{Version Control: CVS, SubVersion, Bazaar, Source Depot, git, TFS}
\ressubitem{Database: MySQL and PostgreSQL, Product Studio}
\ressubitem{Mathematics: Matlab, SciPy, Pylab, Maxima, Mathematica}
\ressubitem{Graphics: Gimp, Inkscape, Blender}
\ressubitem{Office: Microsoft Excel, Word, and Powerpoint, OpenOfice.org, Abiword}
\ressubitem{System installation and administration for Windows (95, 98, XP, Vista, Server 2003 and 2008), Linux (Red Hat, Mandrake, Debian, Ubuntu).}
\ressubitem{Linux web/mail server administration.}
\ressubitem{Home network and firewall administration.}
    \end{reslist}
\resitem{Experience with Scrum}
\resitem{Strong physics and mathematics background}
\resitem{Group coordination and organization}
\resitem{Team player; get along well with others}
\resitem{Quick Learner, adept self teacher}
\resitem{Excellent verbal and written communication skills}
\end{ressection}

\begin{ressection}{TECHNICAL ACCOMPLISHMENTS}
    \begin{ressubsec}{Pyweek 6 - Team Ketchup}{April 2008}
    {Lead Programmer}
    \ressubitem{Pyweek is a Python programming challenge in which entrants to write a game in one week from scratch.  Lead team Ketchup in the creation of ``Bot Builder 2000''. }
    \ressubitem{Entry Page: http://www.pyweek.org/e/Ketchup/}
    \end{ressubsec}
    \begin{ressubsec}{ Balder and Balder2D}{October 2001- Present}
    {Project Administrator/Lead Programmer}
    \ressubitem{Balder and Balder2D are open source zero gravity shooters. Focus shifted from Balder (3D) to Balder2D in the fall of 2004. Both games are written primarily in C++. AI for Balder2D is written in Python. Balder uses the Crystalspace 3D engine for graphics and sound, while Balder2D uses the SDL library. Balder2D is currently version 1.0, release candidate 1. Duties include: majority of design work and implementation for both projects, supervising work done by other team members, and maintaining the project website and releases.}
    \ressubitem{Project Home Page: http://balder.sourceforge.net}
    \end{ressubsec}
\end{ressection}

\begin{ressection}{HOBBIES AND INTERESTS}
\resitem{Soccer, game programming, 3D modeling, experimenting with and contributing to open source software, multitouch computing applications, pottery, drawing, playing games of all sorts (athletic, board, and computer based).}
\end{ressection}

\begin{ressection}{REFERENCES}
    \resitem{Available on request}

\end{ressection}

\end{document}

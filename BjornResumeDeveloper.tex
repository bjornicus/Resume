\documentclass[11pt,oneside]{article}
\usepackage{geometry}
\usepackage{ifthen}
%\usepackage[T1]{fontenc}

\pagestyle{empty}
\geometry{letterpaper,tmargin=0.75in,bmargin=0.75in,lmargin=0.75in,rmargin=0.75in,headheight=0in,headsep=0in,footskip=.3in}

\setlength{\parindent}{0in}
\setlength{\parskip}{0in}
\setlength{\itemsep}{0in}
\setlength{\topsep}{0in}
\setlength{\tabcolsep}{0in}

% Name and contact information
\newcommand{\name}{Bj\o rn Hansen}
\newcommand{\address}{16630 NE 92ND ST\\ Redmond, WA 98052}
\newcommand{\phone}{(425) 283-8374}
\newcommand{\email}{holomorph@gmail.com}

%%%%%%%%%%%%%%%%%%%%%%%%%%%%%%%%%%%%%%%%%%%%%%%%%%%%%%%%%
% New commands and environments

% This defines how the name looks
\newcommand{\bigname}[1]{
    \fontfamily{ppl}\selectfont\huge\scshape#1 \normalsize
}

% A ressection is a main section (<H1>Section</H1>)
\newenvironment{ressection}[1]{
	\vspace{4pt}
	{\fontfamily{phv}\selectfont\Large#1}
	\begin{itemize}
	\vspace{3pt}
}{
	\end{itemize}
}

% A resitem is a simple list element in a ressection (first level)
\newcommand{\resitem}[1]{
	\vspace{-4pt}
	\item \begin{flushleft} #1 \end{flushleft}
}

% A ressubitem is a simple list element in anything but a ressection (second level)
\newcommand{\ressubitem}[1]{
	\vspace{-1pt}
	\item \begin{flushleft} #1 \end{flushleft}
}

% A resbigitem is a complex list element for stuff like jobs and education:
%  Arg 1: Name of company or university
%  Arg 2: Location
%  Arg 3: Title and/or date range
\newcommand{\resbigitem}[3]{
	\vspace{-5pt}
	\item
	\ifthenelse{\equal{#3}{}}
	{\textbf{#1}---#2 } % third argument is empty
	{\textbf{#1}---#2 \\ \textit{#3} } 
}

% This is a list that comes with a resbigitem
\newenvironment{ressubsec}[3]{
	\resbigitem{#1}{#2}{#3}
	\vspace{-2pt}
	\begin{itemize}
}{
	\end{itemize}
}

% This is a simple sublist
\newenvironment{reslist}[1]{
	\resitem{\textbf{#1}}
	\vspace{-5pt}
	\begin{itemize}
}{
	\end{itemize}
}



%%%%%%%%%%%%%%%%%%%%%%%%%%%%%%%%%%%%%%%%%%%%%%%%%%%%%%%%%
% Now for the actual document:
\begin{document}
\begin{center}
\bigname{\name}
\rule{\textwidth}{1pt}
\email \\
\phone \\
\vspace{1pt}
\address
\end{center}

\begin{ressection}{EDUCATION}
    \begin{ressubsec}{University of Washington}{ Seattle, Washington, 2004 - 2007}
    {Master of Science in Aeronautics and Astronautics}
    \ressubitem{Depth Area of Study: Plasma Science}
	\end{ressubsec}
    \begin{ressubsec}{University of Victoria}{Victoria, British Columbia, 2000 - 2004}
    {Bachelor of Science}
    \resitem{Graduated with Distinction in June 2004 with a Major in Physics, and a Minor in Mathematics}
	\end{ressubsec}
\end{ressection}

\begin{ressection}{EXPERIENCE}
    \begin{ressubsec}{Microsoft, XBOX}{Redmond, Washington, August 2008 - Present}
    {Software Developer in Test} 
    \ressubitem{Xbox One Shell Core, 2010 - 2013 \\
    Developed test automation and core automation technology for the Xbox One.  Set up telemetry instrumentation for the overlay shell UI and wrote code to analyze the usage data.
    Developed, as a \lq\lq{}side project\rq\rq{}, a system to enable Beta users to report issues directly from the console and automatically gather supplimental data to help developers understand issues. This became the primary method for filing bugs leading up to, and post, release.  Partnered with developers across the console team to enable them to include their traces and data in this system. }
    \ressubitem{Xbox 360 Foundation Test Tools, 2008 - 2010 \\
    Developed and maintained tools to enable PC driven automated testing of the Xbox 360 shell, including a UI automation library and test harness.  Successfully introduced unit testing and test driven development into my team. \\
    Developed automated tests for Kinect related shell features during the lead-up to initial Kinect launch.  }
    \end{ressubsec}
    \begin{ressubsec}{Volt at Microsoft, XBOX}{Redmond, Washington, May 2008 - August 2008}
    {Software Developmer in Test} 
    \ressubitem{Worked as part of the Xbox Foundation Team on development of the \lq\lq{}Test Case Scheduler\rq\rq{} test harness, which runs automated test cases from Product Studio and auto-files bugs.  Also worked with team members on the UI automation technology.  Additionally responsible for keeping Test Case Scheduler running on the test benches in the lab and following up on issues found during test runs.}
    \end{ressubsec}
    \begin{ressubsec}{Volt at Microsoft, Natural Language Group}{Redmond, Washington, August 2007 - May 2008}
    {Software Developer in Test} 
    \ressubitem{Developed automated tests to verify functionality of proofing tools, primarily the speller engine.}
    \end{ressubsec}
    \begin{ressubsec}{University of Washington, RPPL}{ Seattle, Washington, September 2004 - August, 2007}
    {Research assistant}
    \ressubitem{Worked on an Innovative Confinement Concept device for magnetically confined fusion plasmas at the Redmond
    Plasma Physics Laboratory. Primary projects included design, implementation, and documentation of the glow
    discharge system, heater control system, and the vacuum control system software.}
	\end{ressubsec}
\end{ressection}

%\newpage

\begin{ressection}{SKILLS}
    \begin{reslist}{ Languages}
\ressubitem{C++, C\#, Python, CMD scripts,  Java, Javascript, Perl, SQL}
    \end{reslist}
    \begin{reslist}{ Software}
\ressubitem{Development Environments: Visual Studio, Eclipse, Dev C++, Vim, Sublime}
\ressubitem{Debugging: Visual Studio Debugger, kd, gdb, pdb}
\ressubitem{Testing Frameworks: mstest, Nunit, xUnit, pyunit}
\ressubitem{Version Control: git, TFS, CVS, SubVersion, Bazaar, Source Depot}
\ressubitem{Mathematics: Matlab, SciPy, Pylab, Maxima, Mathematica}
\ressubitem{Graphics: Gimp, Inkscape, Blender}
\ressubitem{System Installation and Administration: Windows, Linux}
    \end{reslist}
  \begin{reslist}{Agile development}
    \ressubitem{Test Driven Development}
    \ressubitem{Pair Programming}
    \ressubitem{Familiar with Scrum, Kanban}
    \end{reslist}
\resitem{Strong physics and mathematics background}
\resitem{Group coordination and organization}
\resitem{Team player; get along well with others}
\resitem{Quick learner, adept self teacher}
\resitem{Excellent verbal and written communication skills}
\end{ressection}

\begin{ressection}{TECHNICAL ACCOMPLISHMENTS}
    \begin{ressubsec}{Pyweek 10  Winner in the Team Category}{March 2010}
    {Lead Programmer}
    \ressubitem{Pyweek is a Python programming challenge in which entrants to write a game in one week from scratch.  I was one of the primary programmers and team organizers for the winning team entry: ``Oscilloscape''.(http://pyweek.org/e/calipygian/) }
    \end{ressubsec}
    \begin{ressubsec}{ Balder and Balder2D}{October 2001- Present}
    {Project Administrator/Lead Programmer}
    \ressubitem{Balder and Balder2D are open source zero gravity shooters. Focus shifted from Balder (3D) to Balder2D in the fall of 2004. Both games are written primarily in C++. AI for Balder2D is written in Python. Balder uses the Crystalspace 3D engine for graphics and sound, while Balder2D uses the SDL library. Balder2D is currently version 1.0, release candidate 1. Duties include: majority of design work and implementation for both projects, supervising work done by other team members, and maintaining the project website and releases.}
    \ressubitem{Project Home Page: http://balder.sourceforge.net}
    \end{ressubsec}
\end{ressection}

\begin{ressection}{HOBBIES AND INTERESTS}
\resitem{Soccer, game programming, 3D modeling, experimenting with and contributing to open source software, multitouch computing applications, pottery, drawing, playing games of all sorts (athletic, board, and computer based).}
\end{ressection}

\begin{ressection}{REFERENCES}
    \resitem{Available on request}

\end{ressection}

\end{document}
